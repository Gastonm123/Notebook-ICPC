%para compilar pdflatex main.text

\documentclass[a4paper,11pt,landscape,twocolumn]{article}
\usepackage[utf8]{inputenc}
\usepackage{amsmath, amssymb, amsthm}
\usepackage{graphicx}
\usepackage{hyperref}
\usepackage{listings}
\usepackage{xcolor}
\usepackage{geometry}
\usepackage[spanish]{babel}

% Aumenta la separación entre columnas
\setlength{\columnsep}{25pt}

\geometry{left=20mm,right=20mm,top=20mm,bottom=20mm}

% Configuración del código en C++
\lstset{
  language=C++,
  basicstyle=\ttfamily\small,
  keywordstyle=\color{blue}\bfseries,
  commentstyle=\color{gray}\itshape,
  stringstyle=\color{red},
  numbers=left,
  numberstyle=\tiny,
  stepnumber=1,
  numbersep=8pt,
  tabsize=4,
  showspaces=false,
  showstringspaces=false,
  breaklines=true,
  breakatwhitespace=true,
  columns=flexible
}

\title{Ayudamemoria}
\author{My Room Is Random Sorted}

\begin{document}

\begin{center}
    \LARGE\textbf{Ayudamemoria}\\[1em]
    \large My room is random Sorted\\[1em]
    \normalsize \today\\[1em]
\end{center}

\tableofcontents

\section{Template}

\lstinputlisting{src/template.cpp}

\subsection{run.sh}

\lstinputlisting{src/run.sh}

\subsection{comp.sh}

\lstinputlisting{src/comp.sh}

\subsection{Makefile}

\lstinputlisting{src/Makefile}

\section{Estructuras de datos}

\subsection{Sparse Table}

\lstinputlisting{src/estructuras/sparse_table.cpp}

\subsection{Segment Tree}

\lstinputlisting{src/estructuras/segment_tree.cpp}

\subsection{Segment Tree Lazy}

\lstinputlisting{src/estructuras/segment_tree_lazy.cpp}

\subsection{Fenwick Tree}

\lstinputlisting{src/estructuras/fenwick.cpp}

\subsection{Tabla Aditiva}

\lstinputlisting{src/estructuras/tablitaAditiva.cpp}

\subsection{Union Find}

\lstinputlisting{src/estructuras/union_find.cpp}

\section{Matemática}

\subsection{Criba Lineal}

\lstinputlisting{src/matematica/criba.cpp}

\subsection{Phollard's Rho}

\lstinputlisting{src/matematica/phollards_rho.cpp}

\subsection{Divisores}

\lstinputlisting{src/matematica/divisores.cpp}

\subsection{Inversos Modulares}

\lstinputlisting{src/matematica/euclides_extendido.cpp}

\lstinputlisting{src/matematica/inversos_modulares.cpp}

\section{Geometria}

\lstinputlisting{src/geometria/geometria.cpp}

\subsection{Lower Envelope}

\lstinputlisting{src/geometria/lowerEnvelope.cpp}

\section{Strings}

\subsection{Hashing}

\lstinputlisting{src/strings/hashing.cpp}

\subsection{Suffix Array}

\lstinputlisting{src/strings/suffix_array.cpp}

\subsection{Kmp}

\lstinputlisting{src/strings/kmp.cpp}

\subsection{Manacher}

\lstinputlisting{src/strings/manacher.cpp}

\subsection{String Functions}

\lstinputlisting{src/strings/string_functions.cpp}

\section{Grafos}

\subsection{Dikjstra}

\lstinputlisting{src/grafos/dikjstra.cpp}

\subsection{LCA}

\lstinputlisting{src/grafos/lca.cpp}

\subsection{Toposort}

\lstinputlisting{src/grafos/toposort.cpp}

\section{Flujo}

\subsection{Dinic}

\lstinputlisting{src/flujo/dinic.cpp}

\subsection{Kuhn}

\lstinputlisting{src/flujo/kuhn.cpp}

\section{Optimización}

\subsection{Ternary Search}

\lstinputlisting{src/optimizacion/ternary_search.cpp}

\subsection{Longest Increasing Subsequence}

\lstinputlisting{src/optimizacion/lis.cpp}

\section{Otros}

\subsection{Mo}

\lstinputlisting{src/otros/mo.cpp}

\subsection{Fijar el numero de decimales}

\lstinputlisting{src/otros/decimales.cpp}

\subsection{Hash Table (Unordered Map/ Unordered Set)}

\lstinputlisting{src/otros/hash_table.cpp}

\subsection{Indexed Set}

\lstinputlisting{src/otros/indexed_set.cpp}

\end{document}
