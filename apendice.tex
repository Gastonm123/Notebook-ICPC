\documentclass[12pt,a4paper]{article}

% Paquetes
\usepackage[utf8]{inputenc}
\usepackage{amsmath, amssymb}
\usepackage{geometry}
\usepackage{array}

% Márgenes cómodos
\geometry{margin=2.5cm}
\setlength{\parindent}{0pt}

\begin{document}

\section*{Apéndice} \vspace{1em}

Dinitz en una red unitaria: $O(\sqrt{V}\cdot E)$ \\


Lista de números con mayor cantidad de divisores hasta $10^n$:\\[0.5em]
(1, 6, 4) (2, 60, 12) (3, 840, 32) (4, 7560, 64) (5, 83160, 128)\\
(6, 720720, 240) (7, 8648640, 448) (8, 73513440, 768) (9, 735134400, 1344)\\
(10, 6983776800, 2304) (11, 97772875200, 4032) (12, 963761198400, 6720)\\
(13, 9316358251200, 10752) (14, 97821761637600, 17280)\\
(15, 866421317361600, 26880) (16, 8086598962041600, 41472) \\
(17, 74801040398884800, 64512) (18, 897612484786617600, 103680)\\

Teorema de Hall: En un grafo bipartito existe un matching perfecto sii para cualquier subconjunto de vertices W, la vecindad de W es mayor o igual que W. $$ |W| \leq |N_G(W)| $$

Teorema de Konig: El numero de aristas en un matching máximo es igual al número de vértices en un cubrimiento por vertices mínimo.\\

Teorema de Dilworth: En todo poset finito, el maximo numero de elementos en una anticadena es igual al tamaño de la minima particion en cadenas del conjunto. \\

Ley de cosenos: Dados dos lados de un triangulo $a, b$ y el ángulo entre ellos $\alpha$, la longitud del otro lado $c$ es: $$ c^2 = a^2 + b^2 - 2ab \cos(\alpha) $$

Ley de senos: En un triángulo la razón, entre cada lado y el seno de su ángulo opuesto, es constante e igual al diámetro de la circunferencia circunscrita.  $$ \frac{a}{sin(\alpha)} = \frac{b}{sin(\beta)} = \frac{c}{\sin(\gamma)} = 2R $$

Valor de $\pi$: $$ \pi = \arccos(-1.0) \quad \text{o} \quad \pi = 4 \cdot \arctan(1.0) $$

Longitud de una cuerda: Sea $\alpha$ el ángulo descripto por una cuerda de longitud $l$ en un círculo de radio $r$.  $$ l = \sqrt{2r^2 \, (1 - \cos(\alpha))} $$ 

Fórmula de Herón: Sea un triángulo con lados $a, b, c$ y semiperímetro $s=\tfrac{a+b+c}{2}$. El área del triángulo es $$A = \sqrt{s(s-a)(s-b)(s-c)}$$

Teorema de Pick: Sean $A$ el área de un polígono, $I$ la cantidad de puntos de coordenadas enteras en su interior, y $B$ la cantidad de puntos de coordenadas enteras en el borde.  $$ A = I + \frac{B}{2} - 1 $$
\end{document}
