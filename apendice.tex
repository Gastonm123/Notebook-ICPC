\documentclass[12pt,a4paper]{article}

% Paquetes
\usepackage[utf8]{inputenc}
\usepackage{amsmath, amssymb}
\usepackage{geometry}
\usepackage{array}

% Márgenes cómodos
\geometry{margin=2.5cm}

\begin{document}

\section*{Apéndice} \vspace{1em}

\begin{itemize}
  \item \textbf{Dinitz en una red unitaria} $O(\sqrt{V}\cdot E)$ \\
  \item \textbf{Ley de cosenos}: Sea un triángulo con lados $A, B, C$ y ángulos 
  $\alpha, \beta, \gamma$ opuestos a $A, B, C$, respectivamente.
  \[
    A^2 = B^2 + C^2 - 2BC \cos(\alpha)
  \]
  \[
    B^2 = A^2 + C^2 - 2AC \cos(\beta)
  \]
  \[
    C^2 = A^2 + B^2 - 2AB \cos(\gamma)
  \]

  \item \textbf{Ley de senos}:
  \[
    \frac{\sin(\alpha)}{A} = \frac{\sin(\beta)}{B} = \frac{\sin(\gamma)}{C}
  \]

  \item \textbf{Valor de $\pi$}: 
  \[
    \pi = \arccos(-1.0) \quad \text{o} \quad \pi = 4 \cdot \arctan(1.0)
  \]

  \item \textbf{Longitud de una cuerda}: Sea $\alpha$ el ángulo descripto por una cuerda de longitud $l$ en un círculo de radio $r$.
  \[
    l = \sqrt{2r^2 \, (1 - \cos(\alpha))}
  \]

  \item \textbf{Fórmula de Herón}: Sea un triángulo con lados $a, b, c$ y semiperímetro $s=\tfrac{a+b+c}{2}$. El área del triángulo es
  \[
    A = \sqrt{s(s-a)(s-b)(s-c)}
  \]

  \item \textbf{Teorema de Pick}: Sean $A$ el área de un polígono, $I$ la cantidad de puntos de coordenadas enteras en su interior, y $B$ la cantidad de puntos de coordenadas enteras en el borde.
  \[
    A = I + \frac{B}{2} - 1
  \]
  \item \textbf{Lista de números con mayor cantidad de divisores hasta $10^n$:}\\[0.5em]
  (1, 6, 4) (2, 60, 12) (3, 840, 32) (4, 7560, 64) (5, 83160, 128)\\
  (6, 720720, 240) (7, 8648640, 448) (8, 73513440, 768) (9, 735134400, 1344)\\
  (10, 6983776800, 2304) (11, 97772875200, 4032) (12, 963761198400, 6720)\\
  (13, 9316358251200, 10752) (14, 97821761637600, 17280)\\
  (15, 866421317361600, 26880) (16, 8086598962041600, 41472) \\
  (17, 74801040398884800, 64512) (18, 897612484786617600, 103680)\\

\end{itemize}

\end{document}
